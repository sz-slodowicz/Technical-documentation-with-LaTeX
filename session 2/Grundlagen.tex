\documentclass[12pt, twocolumn]{article}
\usepackage{blindtext}
\usepackage{graphicx}
\usepackage{hyperref}
\author{Szymon}
\title{Testdokument}
\begin{document}
\maketitle
\blindtext
%Formatierung
\\
The \textbf{quick} \textit{brown} \underline{fox} $_jumps$ $^over$ the lazy dog. 
\\
%Schriftgrösse
{\tiny The} {\scriptsize quick} {\normalsize brown} {\large fox} {\LARGE jumps} {\huge over} the lazy dog.
\\
%Listen
\begin{itemize}
	\item item 1
	\item item 2
\end{itemize}
\blindtext
%Grafiken
\begin{figure}
	\includegraphics{Donald.png}
	\caption{Donald Duck}
	\label{fig:donald}
\end{figure}
\blindtext
%Inlinegrafiken
Das ist ein Beispiel für eine Inlinegrafik, also einer Grafik die im Lauftext eingebunden wird. \includegraphics[height=\baselineskip]{Donald.png}.Der Text geht hier weiter.Die Grafik hat die Höhe des Abstands zwischen zwei Zeilen. 
%Referenzen
See figure \ref{fig:donald}
See the \href{https://ctan.org/pkg/hyperref?lang=de}{documentation of the hyperref} package.

\end{document}

